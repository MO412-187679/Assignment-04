\def\homeworkname{Caley Trees}
\documentclass[assignment = 4]{homework}

\usepackage{caption, subcaption, pdfpages, float}
\usepackage{graphics, wrapfig, pgf, graphicx}
\usepackage{enumitem}


% pacotes para importar código
\usepackage{caption, booktabs}
\usepackage[section, newfloat]{minted}
\definecolor{sepia}{RGB}{252,246,226}
\setminted{
    bgcolor = sepia,
    style   = pastie,
    frame   = leftline,
    autogobble,
    samepage,
    python3,
    breaklines
}
\setmintedinline{
    bgcolor={}
}

% ambientes de códigos de Python
\newmintedfile[pyinclude]{python3}{}
\newmintinline[pyline]{python3}{}
\newcommand{\pyref}[2]{\href{#1}{\texttt{#2}}}

% \SetupFloatingEnvironment{listing}{name=Código}
% \captionsetup[listing]{position=below,skip=-1pt}

\usepackage{csquotes}
\usepackage[
    style    = verbose-ibid,
    autocite = footnote,
    notetype = foot+end,
    backend  = biber
]{biblatex}
\addbibresource{references.bib}
\usepackage[section]{placeins}

\usepackage[hidelinks]{hyperref}
\usepackage[noabbrev, nameinlink]{cleveref}
\hypersetup{
    pdftitle  = {MO412/MC908 - Assignment 4},
    pdfauthor = {Tiago de Paula - 187679}
}

\newcommand{\textref}[2]{
    \hyperref[#2]{#1 \ref*{#2}}
}

\renewcommand{\vec}[1]{\mathbf{#1}}

\DeclareMathOperator{\round}{round}

\usepackage{import, multirow}
\usepackage{pgf, tikz}
\usetikzlibrary{matrix}
\usetikzlibrary{positioning}
\usetikzlibrary{automata}
\usetikzlibrary{shapes}

\usepackage{wrapfig}
\usepackage{booktabs}

\newenvironment{kmatrix}[1][1.3cm]{
    \begin{tikzpicture}[node distance=0cm]
        \tikzset{square matrix/.style={
                matrix of nodes,
                column sep=-\pgflinewidth, row sep=-\pgflinewidth,
                nodes={draw,
                    minimum height=#1,
                    anchor=center,
                    text width=#1,
                    align=center,
                    inner sep=0pt
                },
            },
            square matrix/.default=#1
        }
}{
    \end{tikzpicture}%
}

\newcommand*{\Scale}[2][4]{\scalebox{#1}{\ensuremath{#2}}}%

\newcommand{\red}[1]{\textcolor{red}{\textbf{#1}}}
\def\qm{?}

\begin{document}
    \pagestyle{main}

    Consider Cayley trees $T(k,P)$ as defined in Problem 4 of Barabasi's book Chapter 3 on Random Networks, for $k \geq 3$.

    \begin{enumerate}
        \item Find a formula for the diameter $d_{\max}$ of $T(k,P)$ in terms of $k$ and $P$.
        \item Find a formula for the number of nodes $N$ of $T(k,P)$ in terms of $k$ and $P$.
        \item Does $T(k,P)$ exhibit small world behavior? \\
        To answer that, compute the limit of $d_{\max}/\ln N$ for fixed $k \geq 3$ and $P$ going to infinity.  If this limit is a constant larger than 0 then, yes, $T(k,P)$ has the small world property.
    \end{enumerate}

    \section{Definition}

    A Cayley tree is a symmetric tree, constructed starting from a central node of degree $k$. Each node at distance $d$ from the central node has degree $k$, until we reach the nodes at distance $P$ that have degree one and are called leaves. From \cite{barabasi}.

    \begin{figure}[H]
        \centering
        \def\svgwidth{0.5\textwidth}
        \import{images/}{cayley-tree.tex}

        \caption{Cayley tree with $k = 3$ and $P = 5$.}
    \end{figure}


\end{document}
